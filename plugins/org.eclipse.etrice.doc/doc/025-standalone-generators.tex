\chapter{Standalone Generators}

The eTrice generators can be called from the command line.
This is useful e.g. if they should participate in a build
process driven by command line tools like make.

The generators are distributed as self contained jars and
can be called using

\texttt{jave -jar <the generator.jar> [arguments]}.

The options for the Java generator are

\begin{verbatim}
org.eclipse.etrice.generator.java.Main [-saveGenModel <genmodel path>] [-genDocu] [-lib]
[-noexit] [-saveGenModel <genmodel path>] [-inc] [-genDir <generation directory>]
[-genInfoDir <generation info directory>] [-genDocDir <gen documentation directory>]
[-debug] [-msc_instr] [-gen_as_verbose] [-help] [-persistable]
[-storeDataObj] <list of model file paths>

      <list of model file paths>         # model file paths may be specified as
                                         # e.g. C:\path\to\model\mymodel.room
      -genDocu                           # if specified documentation is created
      -lib                               # if specified all classes are generated
                                         # and no instances
      -noexit                            # if specified the JVM is not exited
      -saveGenModel <genmodel path>      # if specified the generator model will
                                         # be saved to this location
      -inc                               # if specified the generation is incremental
      -genDir <generation directory>     # the directory for generated files
      -genInfoDir <generation info dir>  # the directory for generated info files
      -genDocDir <gen documentation dir> # the directory for generated documentation files
      -debug                             # if specified create debug output
      -msc_instr                         # generate instrumentation for MSC generation
      -gen_as_verbose                    # generate instrumentation for verbose console output
      -help                              # display this help text

      -persistable                       # if specified make actor classes persistable
      -storeDataObj                      # if specified equip actor classes with
                                         # store/restore using POJOs
\end{verbatim}

The options for the C generator are

\begin{verbatim}
org.eclipse.etrice.generator.c.Main [-saveGenModel <genmodel path>] [-genDocu] [-lib]
[-noexit] [-saveGenModel <genmodel path>] [-inc] [-genDir <generation directory>]
[-genInfoDir <generation info directory>] [-genDocDir <gen documentation directory>]
[-debug] [-msc_instr] [-gen_as_verbose] [-help] [-persistable]
[-storeDataObj] <list of model file paths>

      <list of model file paths>         # model file paths may be specified as
                                         # e.g. C:\path\to\model\mymodel.room
      -genDocu                           # if specified documentation is created
      -lib                               # if specified all classes are generated
                                         # and no instances
      -noexit                            # if specified the JVM is not exited
      -saveGenModel <genmodel path>      # if specified the generator model will
                                         # be saved to this location
      -inc                               # if specified the generation is incremental
      -genDir <generation directory>     # the directory for generated files
      -genInfoDir <generation info dir>  # the directory for generated info files
      -genDocDir <gen documentation dir> # the directory for generated documentation files
      -debug                             # if specified create debug output
      -msc_instr                         # generate instrumentation for MSC generation
      -gen_as_verbose                    # generate instrumentation for verbose console output
      -help                              # display this help text
\end{verbatim}
