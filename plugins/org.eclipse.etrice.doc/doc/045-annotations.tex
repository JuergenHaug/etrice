\section{Annotations}

In \eTrice{} it is possible to use annotations similar to Java annotations.
Annotation types can be defined together with their targets and other properties and
later they can be used.

Annotations can be processed by the code generator to influence its behavior.

\subsection{Annotation Type Definitions}

Examples of such definitions can be found in the files \texttt{Annotations.room} which are part of the modellibs.
They contain definitions of annotations that are recognized by the generator of the associated language.

Consider e.g. the annotations definitions for Java

\lstinputlisting[language=ROOM]{../../../runtime/org.eclipse.etrice.modellib.java/model/Annotations.room}

Here we find two definitions. The meaning of those annotations will be explained later in section
\ref{sec:predef_annotations} about "\nameref{sec:predef_annotations}".

The annotation type definition defines a target where the annotation is allowed to be used.
This can be one of

\begin{itemize}
\item DataClass
\item ActorClass
\item ActorBehavior
\item ProtocolClass
\item CompoundProtocolClass
\item SubSystemClass
\item LogicalSystem
\end{itemize}

Attributes can be added as needed and qualified as mandatory or optional.
Attributes have a type (similar as the PrimitiveType but with the understanding that
ptChar is a string). Another attribute type is enum with an explicit list of allowed
enum literals.

\subsection{Usage and Effect of the Pre-defined Annotations}
\label{sec:predef_annotations}

The \eTrice{} generators currently implement two annotations.

\subsubsection{BehaviorManual}

This annotation has no attribute. If specified the code generator won't generate a state machine but
part of the interface and methods of an actor class.

\textbf{Java}

An abstract base class \texttt{Abstract<ActorClassName>} is generated which contains ports, SAPs and attributes as members.
The \texttt{receiveEvent()} method is dispatching to distinct methods per pair of interface item (port or SAP) and message
coming in from this interface item. The user has to sub class the abstract base class and may override the
generated methods as needed.

\textbf{C}

The generator is only generating a public header file and is leaving its implementation to the user.

\subsubsection{ActorBaseClass}

This annotation is defined for Java only. It tells the generator that the generated actor class should
inherit from the specified base class (mandatory string parameters class and package).

If the actor class is modeled as having another actor base class then the annotation has no effect.
